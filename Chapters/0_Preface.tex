% Предисловие

\documentclass[a4paper,12pt]{book}

%%% Работа с русским языком
\usepackage{cmap}					% поиск в PDF
\usepackage[T2A]{fontenc}			% кодировка
\usepackage[utf8]{inputenc}			% кодировка исходного текста
\usepackage[english,russian]{babel}	% локализация и переносы
\usepackage{indentfirst}            % после заголовков ставится абзацный отступ
\usepackage{titlesec}               % Для заголовков
\usepackage{graphicx}               % Для работы с рисунками
\usepackage{underscore}             % Для подчеркиваний
\usepackage{caption}                % Для подписей к фигурам
% Выравнивание заголовка по правому краю
\usepackage{sectsty}
% Использование цвета
\usepackage{xcolor}
% Комментирование блока
\usepackage{comment}

\begin{document}

\frontmatter

\chapterfont{\raggedleft}
% Preface
\chapter{Предисловие}

{
\begin{comment}
No programming technique solves all problems.
No programming language produces only correct results.
No programmer should start each project from scratch.
\end{comment}

\raggedleft
Ни одна техника программирования не решит все проблемы.\\
Ни один язык программирования не даст только верные результаты.\\
Ни один программист не должен начинать каждый проект с нуля.\\
}
\bigskip
{
\begin{comment}
Object-oriented programming is the current cure-all — although it has been
around for much more then ten years. At the core, there is little more to it then
finally applying the good programming principles which we have been taught for
more then twenty years. C++ (Eiffel, Oberon-2, Smalltalk ... take your pick) is the
New Language because it is object-oriented — although you need not use it that
way if you do not want to (or know how to), and it turns out that you can do just as
well with plain ANSI-C. Only object-orientation permits code reuse between projects
— although the idea of subroutines is as old as computers and good programmers
always carried their toolkits and libraries with them.
\end{comment}
Объектно-ориентированное программирование - совеременная панацея несмотря на то, 
что о нём известно уже больше десяти лет. По сути, это нечто большее, чем 
применение хороших принципов программирования, которым нас учили 
на протяжении двух десятилетий. \mbox{C++} (\mbox{Eiffel,} \mbox{Oberon-2,} \mbox{Smalltalk} ... впишите свой вариант) - это 
Новый Язык, потому что он --- объектно-ориентированный, хоть и нет необходимости применять его таким 
образом, если вы не хотите (или не знаете как), и, оказывается, это также относится к чистому ANSI-C. Только объектно-ориентированный подход позволяет обеспечить повторное использование кода в между проектами, 
хоть и идея подпрограмм стара как компьютеры и хорошие программисты 
всегда имеют свои инструменты и библиотеки под рукой.
}

{
\begin{comment}
This book is not going to praise object-oriented programming or condemn the
Old Way. We are simply going to use ANSI-C to discover how object-oriented programming
is done, what its techniques are, why they help us solve bigger problems,
and how we harness generality and program to catch mistakes earlier. Along
the way we encounter all the jargon — classes, inheritance, instances, linkage,
methods, objects, polymorphisms, and more — but we take it out of the realm of
magic and see how it translates into the things we have known and done all along.
\end{comment}
В этой книге не будет восхваления объектно-ориентированного программирования или же порицания 
Старого Способа. Мы только будем использовать ANSI-C, чтобы узнать как применяется объектно-ориентированное программирование, 
каковы его техники, почему они помогают решать большие задачи, 
и как сопрягаем обобщения и программы, чтобы заранее обнаружить ошибки. В 
книге мы столкнемся с жаргоном --- классы, наследование, экземпляр, компоновка, 
методы, объекты, полиморфизм и прочее --- но мы возьмём эти слова из мира 
магии и увидим как они превращаются в вещи, которые мы знаем и постоянно используем.
}

{
\begin{comment}
I had fun discovering that ANSI-C is a full-scale object-oriented language. To
share this fun you need to be reasonably fluent in ANSI-C to begin with — feeling
comfortable with structures, pointers, prototypes, and function pointers is a must.
Working through the book you will encounter all the newspeak — according to
Orwell and Webster a language ‘‘designed to diminish the range of thought’’ — and
I will try to demonstrate how it merely combines all the good programming principles
that you always wanted to employ into a coherent approach. As a result, you
may well become a more proficient ANSI-C programmer.
\end{comment}
Я получал удовольствие, раскрывая ANSI-C как полномасштабный объектно-ориентированный язык. Чтобы
разделить со мной это чувство, прежде всего, вам нужно быть достаточно подкованными в ANSI-C -
вы должны свободно работать со структурами, указателями, прототипами и указателями на функции. 
Прорабатывая книгу вы столкнётесь с новоязом --- согласно 
Оруэллу и Вебстер --- языку, ,,предельно сужающему возможность выразить что-либо``, и 
я постараюсь продемонстрировать как просто это сочетается со всеми практиками хорошего программирования,
которые вы всегда хотели применить в последовательном подходе. В результате вы
вполне можете стать более искусным программистом на ANSI-C.
}

\end{document}